\documentclass[conference]{IEEEtran}
\usepackage[utf8]{inputenc}
\usepackage{amsmath}
\usepackage{graphicx}
\usepackage{xcolor}

\title{Enhancing Career Readiness Skills for Engineering Students with Artificial Intelligence}%\\
%\large Subtitle as needed}

\author{
\IEEEauthorblockN{Authors Name/s per 1st Affiliation}
\IEEEauthorblockA{\textit{line 1 (of Affiliation): dept. name of organization}\\
\textit{line 2: name of organization, acronyms acceptable}\\
\textit{line 3: City, Country}\\
\textit{line 4: e-mail address if desired}}
\and
\IEEEauthorblockN{Authors Name/s per 2nd Affiliation}
\IEEEauthorblockA{\textit{line 1 (of Affiliation): dept. name of organization}\\
\textit{line 2: name of organization, acronyms acceptable}\\
\textit{line 3: City, Country}\\
\textit{line 4: e-mail address if desired}}
\and
\IEEEauthorblockN{Authors Name/s per 2nd Affiliation}
\IEEEauthorblockA{\textit{line 1 (of Affiliation): dept. name of organization}\\
\textit{line 2: name of organization, acronyms acceptable}\\
\textit{line 3: City, Country}\\
\textit{line 4: e-mail address if desired}}
\and
\IEEEauthorblockN{Authors Name/s per 2nd Affiliation}
\IEEEauthorblockA{\textit{line 1 (of Affiliation): dept. name of organization}\\
\textit{line 2: name of organization, acronyms acceptable}\\
\textit{line 3: City, Country}\\
\textit{line 4: e-mail address if desired}}
}

\begin{document}

\maketitle

\begin{abstract}

Career readiness is a vital aspect of engineering education, aligning with the National Association of Colleges and Employers (NACE) core competencies that are essential for students' professional success. Skills such as communication, problem-solving, and teamwork are central to these competencies, preparing students for effective participation in the workforce. The increasing role of Artificial Intelligence (AI) in education presents new opportunities to enhance these skills beyond traditional coursework. AI-powered tools, including adaptive feedback systems and collaborative platforms, offer personalized learning experiences that foster the development of these career competencies in engineering students. This paper explores how AI can be leveraged to support career readiness, arguing that its integration into engineering curricula can significantly enhance students' communication, problem-solving, and teamwork abilities. By examining the growing role of AI in education and its alignment with NACE competencies, this research highlights its transformative potential in preparing students for successful careers in engineering.
\end{abstract}

%%%%%%%%%%%%%%%%%%%%%%%%%%%%%%%%%%%%%%%%%%%%%%%%%%%%%%%%%%%
\begin{IEEEkeywords}
component, formatting, style, styling, insert
\end{IEEEkeywords}

\section{Introduction}

In today's competitive and fast-evolving job market, career readiness has become a fundamental aspect of higher education. For engineering students, developing skills that extend beyond technical knowledge is essential for success in the workplace. The National Association of Colleges and Employers (NACE) has identified eight core competencies that are integral to students' professional development and are essential for their transition from academic settings to the workforce. These competencies include: career and self-development, communication, critical thinking and problem-solving, equity and inclusion, leadership, professionalism, teamwork, and technology management. These competencies are designed to provide students with the necessary skills to thrive in a diverse, technology-driven workplace\cite{nace2025competencies}.

Among these competencies, \textit{communication}, \textit{problem-solving}, and \textit{teamwork} stand out as particularly crucial for engineering students. Effective communication is vital for engineers who must regularly explain complex technical ideas to both technical and non-technical audiences\cite{Rusmiyanto23}. Problem-solving skills are at the core of engineering practice, requiring students to think critically and apply creative solutions to real-world challenges. Teamwork is equally important, as engineers often collaborate in multidisciplinary teams to achieve common goals. These three competencies not only help students perform better academically but also prepare them for the demands of the modern workforce\cite{wef2023}.

Despite their importance, these competencies are often underemphasized in traditional engineering curricula, which tend to prioritize technical knowledge and problem-solving techniques\cite{Rovida22}. This gap presents a challenge for educators who aim to prepare students for successful careers in an increasingly interdisciplinary and AI-driven job market . In recent years, the role of Artificial Intelligence (AI) in education has begun to offer new opportunities to address this gap. AI-powered tools, such as adaptive feedback systems, collaborative platforms, and virtual learning environments, have the potential to enhance students' learning experiences by offering personalized instruction and real-time feedback, helping them develop the skills needed for career success.

%Adaptive learning is a technique to use data-driven instruction to adjust and tailor learning experiences to meet the individual needs of each student. Adaptive learning systems can track data such as student progress, engagement, and performance, and use the data to provide personalized learning experiences.


This paper argues that the integration of AI into engineering curricula can significantly enhance career competencies by providing students with the tools they need to develop these skills in a personalized, efficient, and engaging manner. By leveraging AI, educators can create learning environments that simulate real-world situations, offer instant feedback, and foster the development of essential soft skills alongside technical knowledge. This integration will better prepare engineering students for the challenges they will face in their careers, ensuring that they are equipped with the communication, problem-solving, and teamwork skills demanded by employers. Ultimately, AI tools can not only enhance the technical abilities of students but also cultivate the career competencies required to thrive in an AI-driven workforce.

%%%%%%%%%%%%%%%%%%%%%%%%%%%%%%%%%%%%%%%%5
\section{Alignment with Workforce Needs}
The rapid advancements in technology and automation are reshaping the expectations of employers. According to a report by the World Economic Forum, nearly 50\% of all employees will need reskilling by 2025 due to the integration of AI and digital transformation in the workplace \cite{wef2020future}. The growing reliance on AI-driven decision-making, virtual collaboration, and real-time data analysis has increased the demand for professionals who possess not only technical expertise but also strong communication, problem-solving, and teamwork skills. Employers seek candidates who can effectively collaborate with AI systems, make informed decisions, and contribute meaningfully to complex projects.

By integrating AI-powered learning tools into engineering curricula, educational institutions can better equip students with these essential career competencies. AI has the potential to bridge the gap between theoretical knowledge and practical application, ensuring that graduates are well-prepared to navigate the evolving job market.


\section{Overview of NACE Core Competencies}

The National Association of Colleges and Employers (NACE) has identified eight essential career readiness competencies that are critical for students’ transition from academia to the professional world. These competencies serve as a framework for developing the skills necessary for workplace success across various industries, including engineering. The eight NACE competencies are:

\begin{itemize}
    \item \textbf{Career and Self-Development}: Engaging in continuous learning, self-reflection, and professional growth.
    \item \textbf{Communication}: Clearly and effectively exchanging information with diverse audiences.
    \item \textbf{Critical Thinking and Problem-Solving}: Analyzing issues, making decisions, and overcoming obstacles.
    \item \textbf{Equity and Inclusion}: Demonstrating awareness and inclusivity in diverse environments.
    \item \textbf{Leadership}: Recognizing personal and team strengths to achieve common goals.
    \item \textbf{Professionalism}: Exhibiting ethical behavior, responsibility, and effective work habits.
    \item \textbf{Teamwork}: Collaborating with others to achieve shared objectives.
    \item \textbf{Technology Management}: Adapting to and effectively utilizing technology in the workplace.
\end{itemize}

While all eight competencies are crucial for career readiness, this paper focuses on three key areas—\textbf{communication, problem-solving, and teamwork}—which are particularly important for engineering students. These competencies not only align with the expectations of modern employers but also directly influence engineers’ ability to collaborate, innovate, and contribute effectively to their organizations.

\subsection{Communication Competency}
Communication is a fundamental skill for engineers, as they must convey complex technical concepts to both technical and non-technical audiences. Engineers regularly engage in writing reports, delivering presentations, and collaborating with diverse stakeholders, including clients, project managers, and policymakers. A lack of strong communication skills can lead to misunderstandings, project inefficiencies, and reduced effectiveness in multidisciplinary environments \cite{irish2013engineering}. According to NACE, employers consistently rank communication as one of the top skills they seek in job candidates \cite{nace2025competencies}. In an AI-driven world, tools such as AI-powered writing assistants, speech analysis software, and virtual presentation platforms can help students refine their written and verbal communication skills.
\textcolor{red}{
\begin{itemize}
    \item Sam
    \item Describe AI tools that help improve communication skills (e.g., Grammarly, AI-powered language tutors, or virtual presentation platforms).
    \item Describe how AI could simulate real-world communication challenges (e.g., technical presentations or client interactions).
    \item Several universities are integrating AI-driven tools into their communication training programs. Some institutions use AI-based writing assistants in technical writing courses, while others incorporate virtual reality (VR) and AI simulations for public speaking and professional interactions. Review different universities' approaches.
\end{itemize}
}
\subsection{Problem-Solving Competency}
Problem-solving is at the heart of engineering practice. Engineers are often required to analyze complex challenges, identify solutions, and implement effective strategies. Employers value candidates who can demonstrate critical thinking and adaptability in real-world situations \cite{jonassen2006everyday}. AI can enhance problem-solving skills by providing students with dynamic problem-based learning environments, adaptive learning platforms, and AI-driven simulations that challenge them to develop innovative solutions in real time. Additionally, AI can assist in debugging code, optimizing designs, and automating data analysis, making problem-solving more efficient and data-driven.
\textcolor{red}{
\begin{itemize}
    \item Christina
    \item Explain how AI can present dynamic, real-world problem scenarios (e.g., adaptive coding platforms or engineering simulations).
    \item Highlight tools like AI-powered decision-making models that foster critical thinking.
    \item University Adoption of AI for Problem-Solving: many universities integrate AI-powered problem-solving tools into their engineering curricula. Some institutions use AI-based virtual labs, while others implement AI-driven analytics platforms to train students in data-driven decision-making. Review of how different universities incorporate AI in engineering problem-solving and provide insights into best practices and emerging trends in engineering education.
\end{itemize}
}
\subsection{Teamwork Competency}
Engineering projects typically require collaboration among professionals from various disciplines, making teamwork an essential competency. Effective teamwork involves active listening, conflict resolution, and shared decision-making, all of which contribute to a productive work environment \cite{salas2018teamwork}. Employers prioritize candidates who can demonstrate the ability to work effectively in teams, particularly in remote and hybrid work environments \cite{nace2025competencies}. AI-powered collaboration tools, such as intelligent project management systems and AI-driven feedback mechanisms, can facilitate more effective teamwork by providing real-time assistance, tracking team dynamics, and ensuring balanced participation.

\textcolor{red}{
\begin{itemize}
    \item Faiyhaa
    \item AI-Powered Collaborative Platforms for Team-Based Learning AI-driven tools help engineering students collaborate more effectively, whether in classroom projects or professional settings.
    \item AI Simulating Real-World Teamwork Scenarios AI can provide immersive learning experiences that simulate professional engineering teamwork environments
    \item University Adoption of AI for Teamwork Development Many universities integrate AI-based collaborative tools in engineering education to strengthen teamwork skills. Some institutions use AI-powered peer evaluation platforms to enhance group dynamics, while others incorporate AI-driven virtual labs for remote teamwork experiences. A review of how different universities implement AI for teamwork training can provide insights into effective strategies for preparing engineering students for professional collaboration.
\end{itemize}
}


%\begin{thebibliography}{9}
%    \bibitem{shwom2022engineering} Shwom, B. L., & Snyder, L. G. (2022). \textit{Engineering Communication: From Principles to Practice}. Oxford University Press.
%    \bibitem{nace2025competencies} National Association of Colleges and Employers. (2025). \textit{Career Readiness Competencies}. Retrieved from \url{https://www.naceweb.org/}
%    \bibitem{jonassen2006everyday} Jonassen, D. H. (2006). \textit{Everyday Problem Solving in Engineering: Lessons for Engineering Educators}. Journal of Engineering Education, 95(2), 139-151.
%    \bibitem{salas2018teamwork} Salas, E., Reyes, D. L., & McDaniel, S. H. (2018). \textit{The Science of Teamwork: Progress, Reflections, and the Road Ahead}. American Psychologist, 73(4), 593.
%    \bibitem{wef2020future} World Economic Forum. (2020). \textit{The Future of Jobs Report 2020}. Retrieved from \url{https://www.weforum.org/reports/the-future-of-jobs-report-2020}
%\end{thebibliography}


%%%%%%%%%%%%%%%%%%%%%%%%%%%%%%%%%%%%%%%%%%


%%%%%%%%%%%%%%%%%%%%%%%%%%%%%%%%%%%%5
%\section{Potential Benefits of Integrating AI into Engineering Curricula}
%The integration of artificial intelligence (AI) tools into engineering education offers transformative opportunities to enhance student learning and career readiness. Below, we outline the key benefits:

%\begin{itemize}
%    \item \textbf{Increased Personalization and Adaptability in Learning Experiences:} 
%    AI-powered educational platforms enable personalized learning paths tailored to individual students' strengths, weaknesses, and learning preferences. Adaptive learning systems, such as those used in intelligent tutoring systems (e.g., Carnegie Learning, ALEKS), assess students' performance in real-time and adjust the difficulty level of problems or recommend specific resources to bridge knowledge gaps \cite{holmes2019ai}. This approach ensures that students progress at their own pace while mastering fundamental concepts effectively.
    
%    \item \textbf{Opportunities for Real-Time Feedback and Iterative Improvement:}
%    AI tools offer students immediate and actionable feedback on assignments, projects, and assessments. For instance, AI-driven coding platforms like Codio and HackerRank provide instant debugging suggestions and performance analysis. This iterative feedback loop enables students to refine their solutions, fostering a deeper understanding of technical concepts and improving problem-solving skills. Additionally, such tools save instructors significant time in grading, allowing them to focus on higher-order teaching tasks \cite{WANG2024124167}.
    
%    \item \textbf{Preparing Students for AI-Driven Workplaces:}
%    The rapid adoption of AI technologies across industries has made proficiency in these tools a critical competency for future engineers. Familiarizing students with AI applications—such as natural language processing, machine learning, and predictive analytics—prepares them for roles in AI-driven workplaces. Moreover, exposure to AI tools enhances technical and transferable skills, including critical thinking, collaboration, and technological literacy, which align with NACE's core competencies \cite{wef2020future}. By embedding these tools into engineering curricula, students gain a competitive edge in the job market and develop the confidence to navigate evolving technological landscapes.
%\end{itemize}

%The integration of AI tools into engineering education not only enhances the learning experience but also bridges the gap between academia and industry. By leveraging the capabilities of AI, institutions can equip students with the knowledge, skills, and adaptability required to thrive in a technology-driven world.

%%%%%%%%%%%%%%%%%%%%%%%%%%%%%%%%%%%%%%

\section{Challenges and Considerations}
\begin{itemize}
    \item Acknowledge potential barriers (e.g., lack of access, resistance to change, ethical concerns like data privacy and AI bias).
    \item Propose solutions to mitigate these challenges (e.g., subsidized AI tools, training for educators, ethical AI use policies).
\end{itemize}

%\section{Proposed Framework for Implementation}
%Offer a high-level framework for incorporating AI into engineering curricula:
%\begin{enumerate}
%    \item Step 1: Identify relevant NACE competencies to focus on.
%    \item Step 2: Select or develop AI tools aligned with those competencies.
%    \item Step 3: Design coursework and projects integrating AI tools for skill development.
%    \item Step 4: Evaluate the impact through feedback and reflection exercises.
    
%\end{enumerate}

\section*{Conclusion}
Reiterate the potential of AI to transform career readiness for engineering students.
Emphasize the alignment with NACE competencies.
Call for educators, institutions, and policymakers to embrace this approach.

\subsection{Figures and Tables}
Place figures and tables at the top and bottom of columns...

%\begin{table}[htbp]
%\caption{Table Type Styles}
%\begin{center}
%\begin{tabular}{|c|c|c|c|}
%\hline
%\textbf{Table Head} & \multicolumn{3}{|c|}{\textbf{Table Column Head}} \\
%\cline{2-4} 
% & \textbf{Subhead} & \textbf{Subhead} & \textbf{Subhead} \\
%\hline
%copy & More table copy & & \\
%\hline
%\end{tabular}
%\label{tab1}
%\end{center}
%\end{table}

%\begin{figure}[htbp]
%\centerline{\includegraphics[width=0.5\textwidth]{fig1.png}}
%\caption{Example of a figure caption.}
%\label{fig}
%\end{figure}



\bibliographystyle{IEEEtranS} % Alphabetical IEEE Transaction Style
\bibliography{thesis}

%\begin{thebibliography}{00}
%\bibitem{b1} G. Eason, B. Noble, and I. N. Sneddon, "On certain %integrals of Lipschitz-Hankel type involving products of Bessel functions," Phil. Trans. Roy. Soc. London, vol. A247, pp. 529–551, April 1955.
%\bibitem{b2} J. Clerk Maxwell, A Treatise on Electricity and Magnetism, 3rd ed., vol. 2. Oxford: Clarendon, 1892, pp.68–73.
%\bibitem{b3} I. S. Jacobs and C. P. Bean, "Fine particles, thin films and exchange anisotropy," in Magnetism, vol. III, G. T. Rado and H. Suhl, Eds. New York: Academic, 1963, pp. 271–350.
%\end{thebibliography}

\end{document}


